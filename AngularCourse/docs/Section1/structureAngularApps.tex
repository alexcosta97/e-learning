\chapter{Structure of Angular Apps}
\section{e2e Folder}
e2e stands for End-To-End and this is where we write end-to-end tests for our application.

\section{node modules}
This is where we store all the third-party libraries that our application depends upon. It is only for development, so when the app is compiled, parts of the third-party libraries are put in a bundle and are deployed into the server. The whole folder doesn't go into the server.

\section{src}
This is where the source code of the application resides. Inside this folder we have many folders and files, such as the index.html, that contains our application. In this file there is no references for a JS or CSS file.

Then we have main.ts. This file is the starting point of our application.

The polyfills file is a file that imports some scripts that are required to load angular. This file serves to fill the gaps between the features that Angular needs and the features that are supported by current browsers.

The styles file is where we add the global styles for our application.

Then we have test.ts that is used to set our testing environment.

\subsection{app}
Inside this module we have a module and a component. All projects have at least one module and one component.
\begin{itemize}
    \item app/app.component. : Defines the AppComponent along with an HTML template, CSS stylesheet, and a unit test.
    It is the root component of what will become a tree of nested components as the application evolves
    \item app.module.ts: Defines AppModule, the root component that tells Angular how to assemble the application.
    Right now it declares only the AppComponent. Soon there will be more components to declare.
\end{itemize}

\subsection{assets}
The assets folder is where we store the static assets of our app such as images, text files and icons.

\subsection{environments}
This folder contains one file for each of our destination environments, each exporting simple
configuration variables to use in our application. The files are replaced on-the-fly when we build
our app. We might use a different API endpoint for development than we do for production or maybe different
analytics tokens. We might even use some mock services. Either way, the CLI has us covered.

\section{package.json}
This is a standard file that every Node project has. It has some basic settings such as the name and version of the application. It also lists the dependencies for the application, which determines the libraries that our application is dependent upon. We have many libraries listed, but there are some that we may want to delete or add.

Then there is another section called devDependencies, which are the libraries that are needed to develop the application.

\section{tsconfig.json}
This is the files that has the settings for the TypeScript compiler. It looks at the settings and then compiles the TypeScript code into JS that the browsers can understand.

\section{tslint.json}
This file includes a load of settings for TSLint, which is a static analysis tool for TS code for readability, maintainability and functionality errors.

\section{index.html}
The main HTML page that is served when someone visits the site. Most of the time we'll never need to edit it. The CLI automatically adds all the js and css files when
building our app so we never need to add any script of link tags manually.

\section{main.ts}
The main entry point for our app. Compiles the application with the JIT compiler and
bootstraps the application's root module (AppModule) to run in the browser. We can also use
the AOT compiler without changing any code by appending the --aot flag to the ng build and
ng serve commands.

\section{polyfills.ts}
Different browsers have different levels of support of the web standards. Polyfills help
normalize those differences. We should be pretty safe with core-js and zone.js, but be sure to
check the Browser Support guide for more information.

\section{styles.css}
Our global styles go here. Most of the time we'll want to have local styles in our components
for easier maintenance, but styles that affect all of our app need to be in a central place.

\section{test.ts}
This is the main entry point for our unit tests. It has some custom configuration that might be
unfamiliar, but it's not something we'll need to edit.