\chapter{The Application Shell}
\section{Install the Angular CLI}
In your command line, after installing node.js and
getting npm updated (see its website), install
Angular by using the following command:

$npm install -g @angular/cli$

\section{Create a new application}
Create a new project named angular-tour-of-heroes with
the following CLI command, in the folder in which you
want to create the app directory:

ng new angular-tour-of-heroes

The Angular CLI generated a new project with a default
application and supporting files.

\section{Serve the application}
Go to the project directory in your CLI and launch the application:
\begin{itemize}
    \item cd [project-directory]
    \item ng serve --open
\end{itemize}

The ng serve command build the app, starts the development
server, watches the source files, and rebuilds the app as we
make changes to those files. The --open flag opens a browser
at localhost:4200/

\section{Angular components}
The page we see is the application shell. This shell is
controlled by an Angular component named AppComponent.

Components are the fundamental building blocks of angular
applications. They display data on the screen, listen for
user input, and take action based on that input.

\section{Change the application title}
Open the project in your favourite editor or IDE and navigate to the src/app folder. You'll find an implementation of the shell AppComponent distributed over three files@
\begin{enumerate}
    \item app.component.ts - the component class code, written in TypeScript
    \item app.component.html - the component template, written in HTML
    \item app.component.css - the component's private CSS styles
\end{enumerate}

Open the component class file (app.component.ts) and change the value of the title property to 'Tour of Heroes'.

\begin{lstlisting}{language=JavaScript}
    title = 'Tour of Heroes';
\end{lstlisting}

Open the component template file (app.component.html) and delete the default template generated by the Angular CLI. Replace it with the following line of HTML:

\begin{lstlisting}{language=HTML5}
    <h1>{{title}}</h1>
\end{lstlisting}

The double curly braces are Angular's $interpolation binding$ syntax. This interpolation binding presents the component's title property value inside the HTML header tag. The browser refreshes and displays the new application title.

\section{Add application styles}
Most apps strive for a consistent look across the application. The CLI generated an empty styles.css for this purpose. There is where we shall put our application-wide styles.

Here's an exceprt from the styles.css for the $Tour of Heroes$ sample app:


\begin{lstlisting}{language=CSS}
    /* Application-wide Styles */
    h1{
        color: #369;
        font-family: Arial, Helvetica, sans-serif;
        font-size: 250%;
    }
    h2, h3{
        color: #444;
        font-family: Arial, Helvetica, sans-serif;
        font-weight: ligther;
    }
    body{
        margin: 2em;
    }
    body, input[text], button{
        color: #888;
        font-family: Cambria, Georgia;
    }
    /* everywhere else */
    *{
        font-family: Arial, Helvetica, sans-serif;
    }
\end{lstlisting}

\section{Final code review}
The source code for this tutorial and the complete $Tour of Heroes$ global styles are available in the \href{https://angular.io/generated/zips/toh-pt0/toh-pt0.zip}{download example}.

\section{Summary}
\begin{itemize}
    \item We created the initial application structure using the Angular CLI
    \item We learned that Angular components display data
    \item We used the double curly braces of interpolation to display the app title.
\end{itemize}