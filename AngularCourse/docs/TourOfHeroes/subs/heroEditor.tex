\chapter{The Hero Editor}
The application now has a basic title. Next, we will create a new component to display hero information and place that component in the application shell.

\section{Create the heroes component}
Using the Angular CLI, generate a new component names heroes.

ng generate component heroes

The CLI creates a new folder, src/app/heroes, and generates the three files of the HeroesComponent. The HeroesComponent class file is as follows:

\begin{lstlisting}{language=JavaScript}
    import { Component, OnInit } from '@angular/core';

    @Component({
    selector: 'app-heroes',
    templateUrl: './heroes.component.html',
    styleUrls: ['./heroes.component.css']
    })
    export class HeroesComponent implements OnInit {

    constructor() { }

    ngOnInit() {
    }

    }

\end{lstlisting}

We always import the Component symbol from the Angular core library and annotate the component class with @Component.

@Component is a decorator function that specifies the Angular metadata for the component.
The CLI generated three metadata properties:
\begin{itemize}
    \item selector - the components CSS element selector
    \item templateUrl - the location of the component's template file
    \item styleUrls - the location of the component's private CSS styles.
\end{itemize}

The CSS element selector matches the name of the HTML element that identifies this component within a parent component's template.

The ngOnInit is a lifecycle hook. Angular calls ngOnInit shortly after creating a component. It's a good place to put initialization logic.

Always export the component class so you can important it elsewhere... like in the AppModule.

\section{Add a hero property}
Add a hero property to the Heroes Component for a hero names "Windstorm".

\section{Show the hero}
Open the heroes.component.html template file. Delete the default text generated by the Angular CLI and replace it with a data binding to the new hero property.

\begin{lstlisting}
    {{hero}}
\end{lstlisting}

\section{Show the HeroesComponent view}
To display the HeroesComponent, you must add it to the template of the shell AppComponent.

Remember that app-heroes is the element selector for the HeroesComponent. So add an <app-heroes> element to the AppComponent template file, just below the title.

\begin{lstlisting}
    <h1>{{title}}</h1>
    <app-heroes></app-heroes>
\end{lstlisting}

Assuming that the CLI ng serve command is still running, the browser should refresh and display both the application title and the hero name.

\section{Create a Hero class}
A real hero is more than a name. Create a Hero class in its own file in the src/app folder. Give it id and name properties.

\begin{lstlisting}
    export class Hero{
        id: number;
        name: string;
    }
\end{lstlisting}

Return to the HeroesComponent class and import the Hero class.
Refactor the component's hero property to be of type Hero. Initialize it with an id of 1 and the name Windstorm.
The revised HeroesComponent class file should look like this:

\begin{lstlisting}
    import {Component, OnInit} from '@angular/core';
    import {Hero} from '../hero';

    @Component({
        selector: 'app-heroes',
        templateUrl: './heroes.component.html',
        styleUrls['/heroes.component.css']
    })

    export class HeroesComponent implements OnInit{
        hero: Hero ={
            id: 1,
            name: 'Windstorm'
        }
    };

    constructor(){}

    ngOnInit(){}
\end{lstlisting}