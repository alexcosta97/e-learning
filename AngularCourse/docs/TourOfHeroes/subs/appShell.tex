\chapter{The Application Shell}
\section{Install the Angular CLI}
In your command line, after installing node.js and
getting npm updated (see its website), install
Angular by using the following command:

$npm install -g @angular/cli$

\section{Create a new application}
Create a new project named angular-tour-of-heroes with
the following CLI command, in the folder in which you
want to create the app directory:

ng new angular-tour-of-heroes

The Angular CLI generated a new project with a default
application and supporting files.

\section{Serve the application}
Go to the project directory in your CLI and launch the application:
\begin{itemize}
    \item cd [project-directory]
    \item ng serve --open
\end{itemize}

The ng serve command build the app, starts the development
server, watches the source files, and rebuilds the app as we
make changes to those files. The --open flag opens a browser
at localhost:4200/

\section{Angular components}
The page we see is the application shell. This shell is
controlled by an Angular component named AppComponent.

Components are the fundamental building blocks of angular
applications. They display data on the screen, listen for
user input, and take action based on that input.