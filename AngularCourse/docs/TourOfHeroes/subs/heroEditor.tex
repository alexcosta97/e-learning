\chapter{The Hero Editor}
The application now has a basic title. Next, we will create a new component to display hero information and place that component in the application shell.

\section{Create the heroes component}
Using the Angular CLI, generate a new component names heroes.

ng generate component heroes

The CLI creates a new folder, src/app/heroes, and generates the three files of the HeroesComponent. The HeroesComponent class file is as follows:

\begin{lstlisting}{language=JavaScript}
    import { Component, OnInit } from '@angular/core';

    @Component({
    selector: 'app-heroes',
    templateUrl: './heroes.component.html',
    styleUrls: ['./heroes.component.css']
    })
    export class HeroesComponent implements OnInit {

    constructor() { }

    ngOnInit() {
    }

    }

\end{lstlisting}

We always import the Component symbol from the Angular core library and annotate the component class with @Component.

@Component is a decorator function that specifies the Angular metadata for the component.
The CLI generated three metadata properties:
\begin{itemize}
    \item selector - the components CSS element selector
    \item templateUrl - the location of the component's template file
    \item styleUrls - the location of the component's private CSS styles.
\end{itemize}

The CSS element selector matches the name of the HTML element that identifies this component within a parent component's template.